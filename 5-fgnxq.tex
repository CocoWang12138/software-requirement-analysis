\section{非功能需求}

\subsection{性能需求}

当用户发出操作请求时,系统需要快速、准确地相关操作给与响应。在系统存在空闲资源的情况下,一些简单的系统操作的响应时间不应太长。例如对于简单易行的操作,用户的等待时间不宜过长,系统必须在1秒之内完成操作。对于一些复杂的操作,需要保证操作的正确性,不能出现失误。同时为了应对多数据、多线程任务,系统必须具有稳定性,不能出现数据溢出、系统崩溃的问题。

\subsection{可使用性}

软件性能是指软件系统及时提供相应服务的能力,表现在运行期和开发期两个方面:

\begin{enumerate}
\item 运行期:安全性、鲁棒性优先,易用性、互操作性其次,另需可伸缩性、可靠性。
\item 开发期:可维护性、可测试性优先,可维护性、可重用性其次,另需可扩展性、
易理解性。
\end{enumerate}

\subsection{安全性}

\begin{enumerate}
\item 系统安全。为保证系统的安全性、稳定性。用户只能访问有权限访问的好友空间,只能够实现系统允许的功能,不能够访问到具体的内部数据,也不能实行在权限之外的任何操作。所有用户的相关操作都需要做相应的日志以备查看。
\item 数据加密。对所有用户使用数据用一定算法进行加密,尤其是手机使用数据不能被其它用户随意查看。
\item 数据备份。系统对用户的数据进行备份,以弥补数据的丢失和损坏。
\item 系统日志。本系统应该能够记录系统运行时所发生的所有错误,包括本机错误和
网络错误,这些错误记录便于查找错误的原因。日志同时记录用户的关键性操作信息。
\end{enumerate}

