\section{引言}
\subsection{编写目的}
软件需求规格说明描述了“自然之路”——基于微信小程序的营地自然教育管理程序$1.0$版本,
其软件功能性需求和非功能性需求。这一文档计划由实现和验证系统正确功能的项目团队成员来使用。
除非在其他地方另有说明,这里指定的所有需求都具有高优先级,而且都要在版本$1.0$中加以实现。

\subsection{项目背景}
那丘儿国际森林营地是南外附属幼儿园和江苏航渡教育科技公司联手打造,集教育和文化旅游为一体的综合项目。营地位于南京市江宁区横溪街道,建设规划总面积达2000余亩。
营地集山、水、林、田为一体,秉承生态可持续发展的理念,旨在通过自然和生活的多场景课程与体验,让儿童亲近自然、关爱自然、回归自然,分享自然,实现身心灵能量的提升,增长智慧,拥有决定优质人生的自驱力和生存力。
营地致力于打造能让儿童充分释放天性、激发好奇心和探究欲的原生态沉浸式自然场域,支持儿童探索、学习、发展和成长。
营地自开工建设以来,得到各界支持与关注,已接待来自儿童研究领域的来访专家、学者近千余人次,倍受教育各界人士欢迎。秉承生态可持续发展的理念,旨在通过自然和生活的多场景课程与体验,让儿童亲近自然、关爱自然、回归自然,分享自然,实现身心灵能量的提升,增长智慧,拥有决定优质人生的自驱力和生存力。
充分利用森林营地,来进行自然教育、环境教育的实践,是实施自然教育的重要途径。分析国内外自然教育建设现状,借鉴相关文献资料,笔者发现国外自然教育正向着内容专业化、重视环境伦理培养、开展户外教学的方向发展。并且,自然教育手段更加多元,注重户外教学与课堂教学的结合、有效突出乡土地域特色。与此相比,目前国内自然教育建设方面却存在着诸多问题,首先,自然资源体系尚不能很好地利用,自然教育功能也未能得到有效的发挥。其次,自然教育体系不够完善、教育过程脱离自然环境,再者,自然教育力量薄弱、教育形式单一、教育活动组织不力又缺乏足够的资金和政策支持等。而导致这一现象的主要原因就在于自然之路不够精细的设计、不够系统的管理和运营模式,同时缺乏相应的立法和政策支持。
自然教育应该是那丘儿国际森林营地应该突显的教育职能,也是提升那丘儿国际森林营地旅游品质的重要发展点,更是促进全社会的生态文明发展和建设的有效途径。在我国当前的发展下,欣欣向荣的森林旅游业对自然教育的需求非常急切。
本文将以那丘儿国际森林营地为例,探讨基于微信小程序的营地自然教育管理程序方案。

\subsection{定义}
部分词汇定义见表~\ref{tab:chdyb}。
\begin{table}
\centering
\resizebox{\textwidth}{15mm}{
    \begin{tabular}{|l|l|}
    \hline 词汇名称 & 词汇含义 \\
    \hline 静态数据 & 系统固化在内的描述系统实现功能的一部分数据 \\
    \hline 动态数据 & 在软件运行过程中用户输入的后系统输出给用户的一部分数据,也就是系统要处理的数据 \\
    \hline 数据字典 & 数据字典中的名字都是一些属性与内容的抽象和概括,
    它们的特点是数据的“严密性”和“精确性”\\
    \hline
    \end{tabular}
    }
\caption{
词汇定义表
}
\label{tab:chdyb}
\end{table}

\subsection{文档约定}
使用$LaTex$排版,自定义模板的部分参数如下:

\begin{itemize}
\item 标准页面设置,即 A4 纸型,页边距为:上 2.5CM,下 2.5CM,左 3.5CM,右
2.5CM。
\item 正文为宋体11号字。
\end{itemize}

\subsection{预期的读者和阅读建议}
这一文档计划由实现和验证系统正确功能的项目团队成员或者使用人员(包括软件开发人员、项目经理、营销人员、用户、设计人员、测试人员或者文档的编写人员)来使用。

文档的剩余部分将从系统的综合描述、外部接口需求、系统特性以及其他非功能需求介绍软件需求。

针对不同用户,给出不同用户的阅读关注点:

\begin{itemize}
\item{项目经理:根据该文档了解预期产品的功能,并据此进行系统设计、项目管理。}
\item{设计员:根据需求进一步分析,做出详细具体的系统设计以及数据库的设计。}
\item{测试员:根据本文档测试用例,对软件产品进行功能性测试和非功能性测试。}
\item{用户:了解预期产品的功能和性能,并与分析人员一起对整个需求进行讨论和协
商。}
\end{itemize}

在阅读本文档时,首先要了解产品的功能概貌,然后可以根据自身的需要对每一功能进行适当的了解。

\subsection{参考资料}
\begin{itemize}
\item{毋国庆. 软件需求工程[M]. 机械工业出版社, 2008.}
\item{徐锋. 软件需求最佳实践[M]. 电子工业出版社, 2013.}
\item{(完整版)计算机软件文档编制规范.}
\item{案例-旅游业务申请系统用例图和用例文档}
\item{旅游管理系统需求分析报告}
\end{itemize}



















