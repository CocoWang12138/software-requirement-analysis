\section{Related Work}
\label{section:relatedwork}

We review the related work on Venn diagram, Euler diagram,
region-based set data visualization and line-based set data visualization 
to show the background of our work.
%Sets are ubiquitously used in data analysis.
%It is widely acknowledged that there usually exist several
%intersection relations in a family of sets.
%For the set data that describes multiple literatures,
%it is vital to visualize such relations clearly in survey writing.

\subsection{Venn or Euler Diagrams}

Venn or Euler diagrams are the most classic ways to illustrate set interactions.
Flower et al.~\cite{flower2008euler} describe the characteristics shared by the abstract
Euler diagrams that can be visualized.
%An abstract Euler diagram, according to their work,
%is a formal abstract description of the information at
%is to be displayed as a concrete Euler diagram.
Both missing pieces~\cite{koshman2013metasearch,stapleton2010inductively} and fan diagrams~\cite{kim2007visualizing}
use similar concentric rings layout to visualize three sets.
%The former one uses 3 layers of the rings for pairwise overlaps,
%while the latter one places the overlaps in the outer ring only.
Besides, some approaches try to add glyphs to represent set members intuitively,
such as the work presented by Simonetto et al.~\cite{simonetto2009fully}.
To handle cases where well-matched Euler diagrams~\cite{alsallakh2016state} cannot be drawn,
Simonetto and Auber~\cite{simonetto2008visualise} propose a method,
splitting or duplicating certain sets and subsets into disjoint parts,
connecting these parts using edges.
Nevertheless, large Euler diagrams could be incredibly hard to comprehend.
In this case, different visual metaphors are used to solve this problem.


\subsection{Region-Based and Line-Based Methods}

Region-based methods use a closed curve surrounding the set items to define a region.
In this scenario, colors are encoded to distinguish those sets.
The area of interest can be added to the rendering of classical UML-like diagrams
based on the texture splatting principle~\cite{byelas2006visualization}.
Collins et al.~\cite{collins2009bubble} provide a continuous iso-contour,
i.e. BubbleSets, to depict set membership while keeping the primary layout.
One potential solution is to use field function.
It guarantees that regions of two sets that share no common elements will not cause overlaps~\cite{vihrovs2014inverse}.
%Whereas those colored regions might cause the cognition of the underlying visualization.

When two data items are connected,
the connection could be in many forms~\cite{palmer1994rethinking}, including surfaces mentioned above,
curves, ribbons and so on~\cite{steinberger2011context}.
It has been discussed that elements connected by smooth (curved) lines are
easier to be discriminated~\cite{hoffmann2008evaluating} due to the outstanding
connection and the smooth routing around significant content.
Alper et al. present line sets~\cite{alper2011design},
using a curve to connect all elements of a certain set.
It firstly generates a line to represent each set by
connecting all its members~\cite{lin1973effective}.
Then the lines are drawn as Bezier splines using multiple colors correlated to the sets.
Overlaps are illustrated as concentric rings around the set elements.
Kelp diagrams~\cite{dinkla2012kelp} can solve this problem to some extent.
A nested and a striped style are presented.
Furthermore Kelp Fusion~\cite{meulemans2013kelpfusion} is designed
to generate shortest-path graphs.
However, many of kinds of region-based and line-based tools lack of assistance of embedding the
representative images for each element.
